\documentclass[letterpaper]{scrartcl}	
\usepackage[top=0.88in, bottom=1in, left=1in, right=1in]{geometry}

\makeatletter
\DeclareOldFontCommand{\tt}{\normalfont\ttfamily}{\mathtt}
\makeatother

\usepackage{scalefnt}


%--------------------------------------------------------------
% We need this package, part of the KOMA class, for the custom
% headings.
%--------------------------------------------------------------
\usepackage{scrpage2}	
		

%--------------------------------------------------------------
% One of many packages you can use if you want to include
% graphics.
%--------------------------------------------------------------
\usepackage{graphicx}			

%--------------------------------------------------------------
% The AMS packages are useful but not required. They offer a
% number of nice fonts, environments for formatting multiline
% equations, etc.
%--------------------------------------------------------------
\usepackage{amsmath}			
\usepackage{amsfonts}
\usepackage{amssymb}
\usepackage{amsthm}

%--------------------------------------------------------------
% Basic way to set-up the page margins.
%--------------------------------------------------------------
%\addtolength{\oddsidemargin}{-.2in}
%\addtolength{\evensidemargin}{-.2in}
%\addtolength{\textwidth}{0.45in}
%\addtolength{\topmargin}{-.175in}
%\addtolength{\textheight}{0.75in}

%--------------------------------------------------------------
% Comment out the following to add indents and remove space between paragraphs.
%--------------------------------------------------------------
\usepackage{parskip}

%--------------------------------------------------------------
% This package is used to define custom colours.
%--------------------------------------------------------------
\usepackage{xcolor}

%--------------------------------------------------------------
% A few colours for hyperlinks.
%--------------------------------------------------------------
\definecolor{plum}{rgb}{0.36078, 0.20784, 0.4}
\definecolor{chameleon}{rgb}{0.30588, 0.60392, 0.023529}
\definecolor{cornflower}{rgb}{0.12549, 0.29020, 0.52941}
\definecolor{scarlet}{rgb}{0.8, 0, 0}
\definecolor{brick}{rgb}{0.64314, 0, 0}

%--------------------------------------------------------------
% A command for typesetting and linking an email address.
%--------------------------------------------------------------
\newcommand{\email}[1]{\href{mailto:#1}{\tt \textcolor{cornflower}{#1}}}
\newcommand{\web}[1]{\href{#1}{\tt \textcolor{cornflower}{#1}}}

%--------------------------------------------------------------
%  The following declaration includes the hyperref package and
% assigns metadata. If you compile with pdflatex, this data
% will be automatically included in the pdf file.
%--------------------------------------------------------------
%\usepackage[
%	pdftitle={QFT Tutorial 1},%
%	pdfauthor={PSI Tutors},%
%	pdfsubject={QFT Tutorial 1},%
%	pdfkeywords={PSI},
%	colorlinks=true,
%	linkcolor=cornflower,
%	citecolor=scarlet,
%	urlcolor=chameleon%
%]{hyperref}

%\setcounter{secnumdepth}{2}	% section number depth
%\setcounter{tocdepth}{2}		% depth of TOC

%--------------------------------------------------------------
% Specify the font used in captions.
%--------------------------------------------------------------
\setkomafont{captionlabel}{\usekomafont{descriptionlabel}}

%--------------------------------------------------------------
% This is where we define the custom title. The image that is
% placed on the left-hand-side of the title, PILogo.pdf in
% this case, should be in the same directory as this file. Note
% that you can always use hyperlinks for the Title, Semester,
% and Author fields, below, in case you want to link to a seminar
% web page or a lecturer's email address.
%--------------------------------------------------------------

\titlehead{%
	\vspace*{-1cm}
	\begin{minipage}[b]{4.0cm}
	\includegraphics*[height=1.3cm]{Uniandes_logo.jpeg}%
	\end{minipage}
	\hfill
	\begin{minipage}[b]{12cm}
	\begin{flushright}
		\usekomafont{descriptionlabel}
		\large Machine Learning for Quantum Matter and Technology \\
		\normalsize \normalfont
		J. Carrasquilla, E. Inack, G. Torlai, R. Melko, L. Hayward Sierens
	\end{flushright}
	\end{minipage}
	\\[-3mm]
	\hrule
	\vspace{-3mm}
}
% -----------

%--------------------------------------------------------------
% Other useful physic-related packages
%--------------------------------------------------------------
\usepackage{braket}  
% Use \Bra{}, \Ket{} or \Braket{x | \psi} for Dirac notation

%--------------------------------------------------------------
% Nice numbering for question parts.
%--------------------------------------------------------------
\newcommand{\be}{\begin{equation}}
\newcommand{\ee}{\end{equation}}
\newcommand{\ba}{\begin{eqnarray}}
\newcommand{\ea}{\end{eqnarray}}

\newcommand{\ssk}{\smallskip}

\newcommand{\msk}{\medskip}

\newcommand{\nin}{\noindent}

\newcommand{\beq}{\begin{equation}}

\newcommand{\eeq}{\end{equation}}

\renewcommand{\vec}[1]{{\mathbf{#1}}}
\renewcommand{\labelenumi}{\alph{enumi})}
\renewcommand{\labelenumiii}{\roman{enumiii})}

%%%%%%%%%%%%%

\def\be{\begin{eqnarray}}
\def\ee{\end{eqnarray}}
\newcommand{\nn}{\nonumber}
\newcommand\para{\paragraph{}}
\newcommand{\ft}[2]{{\textstyle\frac{#1}{#2}}}
\newcommand{\eqn}[1]{(\ref{#1})}
\newcommand{\pl}[1]{\frac{\partial {\cal L}}{\partial{#1}}}
\newcommand{\ppp}[2]{\frac{\partial {#1}}{\partial {#2}}}
\newcommand{\ph}[1]{\frac{\partial {\cal H}}{\partial{#1}}}
\newcommand{\leftp}[3]{\left.\ppp{#1}{#2}\right|_{#3}}
%\newcommand{\Vec}[2]{\left(\begin{array}{c} {#1} \\ {#2}\end{array}\right)}
\newcommand\vx{\vec{x}}
\newcommand\vy{\vec{y}}
\newcommand\vp{\vec{p}}
\newcommand\vq{\vec{q}}
\newcommand\vk{\vec{k}}
\newcommand\avp{a^{\ }_{\vp}}
\newcommand\advp{a^\dagger_{\vp}}
\newcommand\ad{a^\dagger}

\newcommand\balpha{\mbox{\boldmath $\alpha$}}
\newcommand\bbeta{\mbox{\boldmath $\beta$}}
\newcommand\bgamma{\mbox{\boldmath $\gamma$}}
\newcommand\bomega{\mbox{\boldmath $\omega$}}
\newcommand\blambda{\mbox{\boldmath $\lambda$}}
\newcommand\bmu{\mbox{\boldmath $\mu$}}
\newcommand\bphi{\mbox{\boldmath $\phi$}}
\newcommand\bzeta{\mbox{\boldmath $\zeta$}}
\newcommand\bsigma{\mbox{\boldmath $\sigma$}}
\newcommand\bepsilon{\mbox{\boldmath $\epsilon$}}
\newcommand\btau{\mbox{\boldmath $\tau$}}
\newcommand\beeta{\mbox{\boldmath $\eta$}}
\newcommand\btheta{\mbox{\boldmath $\theta$}}

\def\norm#1{:\!\!#1\!\!:}

\def\part{\partial}

\def\dbox{\hbox{{$\sqcup$}\llap{$\sqcap$}}}

\def\sla#1{\hbox{{$#1$}\llap{$/$}}}
\def\Dslash{\,\,{\raise.15ex\hbox{/}\mkern-13mu D}}
\def\Dbarslash{\,\,{\raise.15ex\hbox{/}\mkern-12mu {\bar D}}}
\def\delslash{\,\,{\raise.15ex\hbox{/}\mkern-10mu \partial}}
\def\delbarslash{\,\,{\raise.15ex\hbox{/}\mkern-9mu {\bar\partial}}}
\def\pslash{\,\,{\raise.15ex\hbox{/}\mkern-11mu p}}
\def\qslash{\,\,{\raise.15ex\hbox{/}\mkern-9mu q}}
\def\kslash{\,\,{\raise.15ex\hbox{/}\mkern-11mu k}}
\def\eslash{\,\,{\raise.15ex\hbox{/}\mkern-9mu \epsilon}}
\def\calDslash{\,\,{\rais.15ex\hbox{/}\mkern-12mu {\cal D}}}
\newcommand{\slsh}[1]{\,\,{\raise.15ex\hbox{/}\mkern-12mu {#1}}}


\newcommand\Bprime{B${}^\prime$}
\newcommand{\sign}{{\rm sign}}

\newcommand\bx{{\bf x}}
\newcommand\br{{\bf r}}
\newcommand\bF{{\bf F}}
\newcommand\bp{{\bf p}}
\newcommand\bL{{\bf L}}
\newcommand\bR{{\bf R}}
\newcommand\bP{{\bf P}}
\newcommand\bE{{\bf E}}
\newcommand\bB{{\bf B}}
\newcommand\bA{{\bf A}}
\newcommand\bee{{\bf e}}
\newcommand\bte{\tilde{\bf e}}
\def\ket#1{\left| #1 \right\rangle}
\def\bra#1{\left\langle #1 \right|}
\def\vev#1{\left\langle #1 \right\rangle}

\newcommand\lmn[2]{\Lambda^{#1}_{\ #2}}
\newcommand\mup[2]{\eta^{#1 #2}}
\newcommand\mdown[2]{\eta_{#1 #2}}
\newcommand\deld[2]{\delta^{#1}_{#2}}
\newcommand\df{\Delta_F}
\newcommand\cL{{\cal L}}
%\def\theequation{\thesection.\arabic{equation}
%%%%%%%%%

%\renewcommand{\ttdefault}{pcr}

\usepackage{enumitem}

\begin{document}

%\scalefont{1.35}

\title{Tutorial 1: \\ Monte Carlo Simulation of the 2D Ising Model}

\date{May 27, 2019}


\maketitle


In this tutorial, we will study the phase transition in the classical two-dimensional Ising model, with Hamiltonian
\begin{equation*}
H = -J \sum_{\langle ij \rangle} \sigma_i \sigma_j, 
\end{equation*}
where $\sigma_i = \pm 1$, $J$ is the coupling strength and $\sum_{\langle ij \rangle}$ denotes a sum over nearest neighbours.
We will consider simulations on a square lattice with periodic boundaries.
In the thermodynamic limit, the critical temperature is known to be ${ T_\text{c} }/{J} \approx 2.269$.

We will use and modify the two Python programs \texttt{ising{\textunderscore}mc.py} and \texttt{plot{\textunderscore}ising.py} throughout this tutorial 
in order to implement Monte Carlo (MC) methods that estimate $T_\text{c}$ and compare with this known exact solution.
%The program \texttt{plot{\textunderscore}ising.py} reads in these measurements and plots observable quantities as a function of temperature $T$.

%%%%%%%%%%%%%%%%%%%%%%%%%%%%%%
%%%%%%%%%%%%%% Q1 %%%%%%%%%%%%%%
%%%%%%%%%%%%%%%%%%%%%%%%%%%%%%
\section{Monte Carlo algorithm}
Consider the Monte Carlo program \texttt{ising{\textunderscore}mc.py}, which is designed to perform a Monte Carlo simulation (using the single-spin-flip Metropolis algorithm) and record measurements of the system's energy $E$ and magnetization $M$. 

\begin{enumerate}[label=\alph*)]

%%%%%%%%%%%%%% (a) %%%%%%%%%%%%%%
\item Examine the section of the code that computes the two-dimensional \texttt{neighbours} array, which is used when calculating the system's energy. 
The code is already written such that \texttt{neighbours[i,0]} and \texttt{neighbours[i,1]} 
store the lattice location of spin \texttt{i}'s rightward and upward neighbours, respectively.
Modify the code such that it will also store spin \texttt{i}'s leftward neighbour in \texttt{neighbours[i,2]} and its downward neighbour in \texttt{neighbours[i,3]}.

\textbf{Hint:} Don't forget to consider the periodic boundary conditions.

%%%%%%%%%%%%%% (b) %%%%%%%%%%%%%%
\item Examine the \texttt{sweep()} function, which proposes a number \texttt{N{\textunderscore}spins} single spin-flip Monte Carlo updates.
Convince yourself that this code is implementing the single-spin-flip Metropolis algorithm.

%%%%%%%%%%%%%% (c) %%%%%%%%%%%%%%
\item Implement a more efficient way of calculating the energy difference \texttt{deltaE} within the \texttt{sweep()} function. 
The given implementation calculates this energy difference by using the \texttt{getEnergy()} function, 
which involves iterating a loop \texttt{N{\textunderscore}spins} times.
However, you should be able to calculate \texttt{deltaE} by summing only four terms.

%\textbf{Hint:} To appreciate the difference in time required for these two different implementations, 
%set \texttt{animate = False} and increase the linear size \texttt{L} when you run the two versions of the code.

%%%%%%%%%%%%%%% (d) %%%%%%%%%%%%%%
%\item Run the code for various values of \texttt{L}, \texttt{J}, \texttt{n{\textunderscore}eqSweeps} and \texttt{n{\textunderscore}measSweeps} 
%and explain how each of these parameters affects the resulting animated samples.

%%%%%%%%%%%%%% (e) %%%%%%%%%%%%%%
\item Run your code with % \texttt{L=4}, \texttt{J=1}, \texttt{T{\textunderscore}list = np.linspace(5.0,0.5,10)}, 
\texttt{n{\textunderscore}eqSweeps=1000} and \texttt{n{\textunderscore}measSweeps=10000}.
The code will generate files in a directory called \texttt{Data} that will store the energy and magnetization 
corresponding to each of your sampled system configurations.
(In Question 2, we will analyze and plot the resulting data.)

\textbf{Hint:} Set \texttt{animate = False} for this part so that the code runs faster.

\end{enumerate}

%%%%%%%%%%%%%%%%%%%%%%%%%%%%%%
%%%%%%%%%%%%%% Q2 %%%%%%%%%%%%%%
%%%%%%%%%%%%%%%%%%%%%%%%%%%%%%
\section{Estimating the critical temperature}

Recall from lecture that the specific heat $C_V$ and susceptibility $\chi$ can be expressed as 
\begin{equation*}
C_V = \frac{\langle E^2 \rangle - \langle E \rangle^2}{T^2} \, , \qquad \qquad
\chi = \frac{\langle M^2 \rangle - \langle M \rangle^2}{T},
\end{equation*}
where $E$ is the energy and $M = \sum_i \sigma_i$ is the magnetization.
For our Monte Carlo calculations on finite lattices, there is no spontaneous symmetry breaking and therefore $\langle M \rangle = 0$ at all $T$. 
As a result, we instead examine $\langle | M | \rangle$ and calculate the susceptibility as
\begin{equation*}
\chi = \frac{\langle M^2 \rangle - \langle | M | \rangle^2}{T}.
\end{equation*}

The quantities $C_V/N$ versus $T$ and $\chi/N$ versus $T$ both diverge at the critical temperature $T_\text{c}$ in the thermodynamic limit $N \to \infty$.
On a finite lattice, these quantities do not diverge but will acquire peaks near $T_\text{c}$. 

\begin{enumerate}[label=\alph*)]

%%%%%%%%%%%%%% (a) %%%%%%%%%%%%%%
\item Use the code \texttt{plot{\textunderscore}ising.py} to read in the Monte Carlo data from Question 1e 
and plot the estimators for $\langle E \rangle/N$ and $\langle | M | \rangle/N$.
Consider the values you find for these estimators in the limit of small $T$; do they match your theoretical expectations?

%%%%%%%%%%%%%% (b) %%%%%%%%%%%%%%
\item Modify \texttt{plot{\textunderscore}ising.py} to calculate $C_V$ and $\chi$. 
Plot $C_V/N$ and $\chi/N$ versus $T$ and verify that there are peaks in these quantities near $T_\text{c}$.

%%%%%%%%%%%%%% (c) %%%%%%%%%%%%%%
\item Use \texttt{ising{\textunderscore}mc.py} to generate additional data for higher $L$ and for more temperatures close to $T_\text{c}$. 
Modify \texttt{plot{\textunderscore}ising.py} to plot your results for several different values of $L$ and confirm that the peaks in $C_V/N$ and $\chi/N$
get closer to $T_\text{c}$ as $L$ increases.

\end{enumerate}
\end{document}
