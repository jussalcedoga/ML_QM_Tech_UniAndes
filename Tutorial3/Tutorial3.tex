\documentclass[letterpaper]{scrartcl}	
\usepackage[top=0.8in, bottom=1in, left=0.9in, right=0.9in]{geometry}

\makeatletter
\DeclareOldFontCommand{\tt}{\normalfont\ttfamily}{\mathtt}
\makeatother

\usepackage{url}
\usepackage{scalefnt}
\usepackage{bm}
\usepackage{cancel}

%--------------------------------------------------------------
% We need this package, part of the KOMA class, for the custom
% headings.
%--------------------------------------------------------------
\usepackage{scrpage2}	
		

%--------------------------------------------------------------
% One of many packages you can use if you want to include
% graphics.
%--------------------------------------------------------------
\usepackage{graphicx}			

%--------------------------------------------------------------
% The AMS packages are useful but not required. They offer a
% number of nice fonts, environments for formatting multiline
% equations, etc.
%--------------------------------------------------------------
\usepackage{amsmath}			
\usepackage{amsfonts}
\usepackage{amssymb}
\usepackage{amsthm}

%--------------------------------------------------------------
% Basic way to set-up the page margins.
%--------------------------------------------------------------
%\addtolength{\oddsidemargin}{-.2in}
%\addtolength{\evensidemargin}{-.2in}
%\addtolength{\textwidth}{0.45in}
%\addtolength{\topmargin}{-.175in}
%\addtolength{\textheight}{0.75in}

%--------------------------------------------------------------
% Comment out the following to add indents and remove space between paragraphs.
%--------------------------------------------------------------
\usepackage{parskip}

%--------------------------------------------------------------
% This package is used to define custom colours.
%--------------------------------------------------------------
\usepackage[usenames,dvipsnames,svgnames,table]{xcolor}

%--------------------------------------------------------------
% Package for adding in solutions:
%--------------------------------------------------------------
\usepackage[nosoln,regf,nolf]{optional}
%\usepackage[soln,regf]{optional}

%\newcommand{\soln}[1]{\opt{soln}{\\[4pt] \textcolor{JungleGreen}{\textbf{Solution:}} #1}}
\newcommand{\soln}[1]{\opt{soln}{\textcolor{JungleGreen}{\usekomafont{descriptionlabel}{Solution:}} #1}}

\newcommand{\hint}[1]{{\usekomafont{descriptionlabel}{Hint:}} #1}
\newcommand{\note}[1]{{\usekomafont{descriptionlabel}{Note:}} #1}
\newcommand{\reference}[1]{{\usekomafont{descriptionlabel}{Reference:}} #1}

%--------------------------------------------------------------
% A few colours for hyperlinks.
%--------------------------------------------------------------
\definecolor{plum}{rgb}{0.36078, 0.20784, 0.4}
\definecolor{chameleon}{rgb}{0.30588, 0.60392, 0.023529}
\definecolor{cornflower}{rgb}{0.12549, 0.29020, 0.52941}
\definecolor{scarlet}{rgb}{0.8, 0, 0}
\definecolor{brick}{rgb}{0.64314, 0, 0}

%--------------------------------------------------------------
% A command for typesetting and linking an email address.
%--------------------------------------------------------------
\newcommand{\email}[1]{\href{mailto:#1}{\tt \textcolor{cornflower}{#1}}}
\newcommand{\web}[1]{\href{#1}{\tt \textcolor{cornflower}{#1}}}

%--------------------------------------------------------------
%  The following declaration includes the hyperref package and
% assigns metadata. If you compile with pdflatex, this data
% will be automatically included in the pdf file.
%--------------------------------------------------------------
%\usepackage[
%	pdftitle={QFT Tutorial 1},%
%	pdfauthor={PSI Tutors},%
%	pdfsubject={QFT Tutorial 1},%
%	pdfkeywords={PSI},
%	colorlinks=true,
%	linkcolor=cornflower,
%	citecolor=scarlet,
%	urlcolor=chameleon%
%]{hyperref}

%\setcounter{secnumdepth}{2}	% section number depth
%\setcounter{tocdepth}{2}		% depth of TOC

%--------------------------------------------------------------
% Specify the font used in captions.
%--------------------------------------------------------------
\setkomafont{captionlabel}{\usekomafont{descriptionlabel}}

%--------------------------------------------------------------
% This is where we define the custom title. The image that is
% placed on the left-hand-side of the title, PILogo.pdf in
% this case, should be in the same directory as this file. Note
% that you can always use hyperlinks for the Title, Semester,
% and Author fields, below, in case you want to link to a seminar
% web page or a lecturer's email address.
%--------------------------------------------------------------

\titlehead{%
	\vspace*{-1cm}
	\begin{minipage}[b]{4.0cm}
	\includegraphics*[height=1.3cm]{Uniandes_logo.jpeg}%
	\end{minipage}
	\hfill
	\begin{minipage}[b]{12cm}
	\begin{flushright}
		\usekomafont{descriptionlabel}
		\large Machine Learning for Quantum Matter and Technology \\
		\normalsize \normalfont
		J. Carrasquilla, E. Inack, G. Torlai, R. Melko, L. Hayward Sierens
	\end{flushright}
	\end{minipage}
	\\[-3mm]
	\hrule
	\vspace{-3mm}
}
% -----------

%--------------------------------------------------------------
% Other useful physic-related packages
%--------------------------------------------------------------
\usepackage{braket}  
% Use \Bra{}, \Ket{} or \Braket{x | \psi} for Dirac notation

%--------------------------------------------------------------
% Nice numbering for question parts.
%--------------------------------------------------------------
\newcommand{\ba}{\begin{eqnarray}}
\newcommand{\ea}{\end{eqnarray}}

\newcommand{\ssk}{\smallskip}
\newcommand{\msk}{\medskip}

\newcommand{\nin}{\noindent}

\newcommand{\beq}{\begin{equation}}
\newcommand{\eeq}{\end{equation}}

\newcommand{\beqs}{\begin{equation*}}
\newcommand{\eeqs}{\end{equation*}}

\renewcommand{\vec}[1]{{\mathbf{#1}}}
\renewcommand{\labelenumi}{\alph{enumi})}
\renewcommand{\labelenumiii}{\roman{enumiii})}

%%%%%%%%%%%%%

\def\be{\begin{eqnarray}}
\def\ee{\end{eqnarray}}
\newcommand{\nn}{\nonumber}
\newcommand\para{\paragraph{}}
\newcommand{\ft}[2]{{\textstyle\frac{#1}{#2}}}
\newcommand{\eqn}[1]{(\ref{#1})}
\newcommand{\pl}[1]{\frac{\partial {\cal L}}{\partial{#1}}}
\newcommand{\ppp}[2]{\frac{\partial {#1}}{\partial {#2}}}
\newcommand{\ph}[1]{\frac{\partial {\cal H}}{\partial{#1}}}
\newcommand{\leftp}[3]{\left.\ppp{#1}{#2}\right|_{#3}}
%\newcommand{\Vec}[2]{\left(\begin{array}{c} {#1} \\ {#2}\end{array}\right)}
\newcommand\vx{\vec{x}}
\newcommand\vy{\vec{y}}
\newcommand\vp{\vec{p}}
\newcommand\vq{\vec{q}}
\newcommand\vk{\vec{k}}
\newcommand\avp{a^{\ }_{\vp}}
\newcommand\advp{a^\dagger_{\vp}}
\newcommand\ad{a^\dagger}

\newcommand\balpha{\mbox{\boldmath $\alpha$}}
\newcommand\bbeta{\mbox{\boldmath $\beta$}}
\newcommand\bgamma{\mbox{\boldmath $\gamma$}}
\newcommand\bomega{\mbox{\boldmath $\omega$}}
\newcommand\blambda{\mbox{\boldmath $\lambda$}}
\newcommand\bmu{\mbox{\boldmath $\mu$}}
\newcommand\bphi{\mbox{\boldmath $\phi$}}
\newcommand\bzeta{\mbox{\boldmath $\zeta$}}
\newcommand\bsigma{\mbox{\boldmath $\sigma$}}
\newcommand\bepsilon{\mbox{\boldmath $\epsilon$}}
\newcommand\btau{\mbox{\boldmath $\tau$}}
\newcommand\beeta{\mbox{\boldmath $\eta$}}
\newcommand\btheta{\mbox{\boldmath $\theta$}}

\def\norm#1{:\!\!#1\!\!:}

\def\part{\partial}

\def\dbox{\hbox{{$\sqcup$}\llap{$\sqcap$}}}

\def\sla#1{\hbox{{$#1$}\llap{$/$}}}
\def\Dslash{\,\,{\raise.15ex\hbox{/}\mkern-13mu D}}
\def\Dbarslash{\,\,{\raise.15ex\hbox{/}\mkern-12mu {\bar D}}}
\def\delslash{\,\,{\raise.15ex\hbox{/}\mkern-10mu \partial}}
\def\delbarslash{\,\,{\raise.15ex\hbox{/}\mkern-9mu {\bar\partial}}}
\def\pslash{\,\,{\raise.15ex\hbox{/}\mkern-11mu p}}
\def\qslash{\,\,{\raise.15ex\hbox{/}\mkern-9mu q}}
\def\kslash{\,\,{\raise.15ex\hbox{/}\mkern-11mu k}}
\def\eslash{\,\,{\raise.15ex\hbox{/}\mkern-9mu \epsilon}}
\def\calDslash{\,\,{\rais.15ex\hbox{/}\mkern-12mu {\cal D}}}
\newcommand{\slsh}[1]{\,\,{\raise.15ex\hbox{/}\mkern-12mu {#1}}}


\newcommand\Bprime{B${}^\prime$}
%\newcommand{\sign}{{\rm sign}}

\newcommand\bx{{\bf x}}
\newcommand\br{{\bf r}}
\newcommand\bF{{\bf F}}
\newcommand\bp{{\bf p}}
\newcommand\bL{{\bf L}}
\newcommand\bR{{\bf R}}
\newcommand\bP{{\bf P}}
\newcommand\bE{{\bf E}}
\newcommand\bB{{\bf B}}
\newcommand\bA{{\bf A}}
\newcommand\bee{{\bf e}}
\newcommand\bte{\tilde{\bf e}}
\def\ket#1{\left| #1 \right\rangle}
\def\bra#1{\left\langle #1 \right|}
\def\vev#1{\left\langle #1 \right\rangle}

\newcommand\lmn[2]{\Lambda^{#1}_{\ #2}}
\newcommand\mup[2]{\eta^{#1 #2}}
\newcommand\mdown[2]{\eta_{#1 #2}}
\newcommand\deld[2]{\delta^{#1}_{#2}}
\newcommand\df{\Delta_F}
\newcommand\cL{{\cal L}}
%\def\theequation{\thesection.\arabic{equation}

\newcounter{solneqn}
%\newcommand{\mytag}{\refstepcounter{equation}\tag{\roman{equationn}}}
\newcommand{\mytag}{\refstepcounter{solneqn}\tag{S.\arabic{solneqn}}}

\newcommand{\appropto}{\mathrel{\vcenter{
  \offinterlineskip\halign{\hfil$##$\cr
    \propto\cr\noalign{\kern2pt}\sim\cr\noalign{\kern-2pt}}}}}
%%%%%%%%%


\DeclareMathOperator{\Tr}{Tr}
\DeclareMathOperator{\sign}{sign}

%\renewcommand{\ttdefault}{pcr}

\usepackage{enumitem}

\begin{document}

%\scalefont{1.35}

\vspace{-3cm}

\opt{nosoln}{\title{Tutorial 3: \\Identifying phase transitions using \\principal component analysis \vspace*{-6mm}}}
\opt{soln}{\title{Tutorial 3 \textcolor{JungleGreen}{Solutions}: \\Identifying phase transitions using \\principal component analysis \vspace*{-6mm}}}

\date{May 29, 2019}

\maketitle

The objective of this tutorial is to use the dimensional reduction technique known as principal component analysis (PCA) to identify phases without explicitly training with phase labels.
You will reproduce the results in Figures 1 and 2 of Reference~\cite{wang}.

The goal of dimensional reduction is to generate a lower-dimensional representation $\mathcal{D}' = \{ \mathbf{x}' \}$ 
of a high-dimensional dataset $\mathcal{D} = \{ \mathbf{x} \}$, where $\mathbf{x}' \in \mathbb{R}^{N'}$, $\mathbf{x} \in \mathbb{R}^{N}$ and $N' < N$.
The lower-dimensional dataset should still encode the important features of the original higher-dimensional data.
The PCA method attempts to accomplish this goal by applying a linear transformation.
In this tutorial we will apply PCA to $N$-dimensional spin configurations of the two-dimensional Ising model.
%In Homework 2, you will consider the \emph{non-linear} $t$-distributed stochastic neighbour embedding (t-SNE) method for visualizing the phases of such models.

Our data is stored in an $M \times N$ matrix $X$, where each of the $M$ rows stores a spin configuration for a system with $N$ spins.
For the two-dimensional Ising model we have $N=L^2$. %while for the two-dimensional Ising lattice gauge theory we have $N=2L^2$.

PCA can be performed on a matrix $X^c$ where each column has mean 0.
One can calculate $X^{c}$ from $X$ as
\begin{equation}
X^c_{ij} = X_{ij} - \frac{1}{M} \sum_{k=1}^M X_{kj}.
\end{equation} 

The principal components $x'_1, x'_2, \ldots$ are then stored in the columns of an $M \times N$ matrix 
\begin{equation}
X' = X_c P,
\end{equation} 
where $P$ is an $N \times N$ matrix. 
$P$ is determined by solving the eigenvalue problem
\begin{equation}
\frac{1}{M-1}X_c^T X_c = P^T D P,
\end{equation}
where $D$ is a diagonal matrix with non-negative entries $\lambda_1 \geq \lambda_2 \geq \cdots \geq \lambda_N \geq 0$.

Another important definition is the so-called \emph{explained variance ratio} $r_\ell$, which measures how much of the variance in the dataset $X$ 
can be explained by the principal component $x'_\ell$. 
This ratio is defined in terms of the eigenvalues $\lambda_\ell$ as
\begin{equation}
r_\ell = \frac{\lambda_\ell}{ \sum_{i=1}^N \lambda_i}.
\end{equation}

For this tutorial, you have been given a dataset containing rows of spin configurations 
$[s_1, s_2, \ldots, s_{N}]$ 
for the two-dimensional Ising model on various sized lattices.
Each spin `up' is stored as 1 and each spin `down' is stored as -1.

You have been given data for $L=20$, 40 and 80. 
Each spin configuration file contains 100 spin configurations at each of the 20 temperatures $T/J = 1.0, 1.1, 1.2, \ldots, 2.9$ such that $M=2000$ for each lattice size.
For each $L$, there is a corresponding file storing the temperature at which each configuration was generated 
(using Monte Carlo simulation).
The temperature data will not be used to determine the principal components and will only be used for data visualization purposes.
If you wish, you could generate this data yourself using the code from Tutorial 1.

%For the Ising gauge theory, you have again been given data for $L=20$, 40 and 80.
%Each spin configuration file contains 500 spin configurations corresponding to $T=0$ (where the system becomes topologically ordered) 
%and 500 spin configurations corresponding to $T=\infty$.
%For each lattice size, there is a corresponding file of labels indicating whether a configuration corresponds to $T=0$ (label 0) 
%or $T=\infty$ (label 1).
%Similar to the temperatures in the Ising model, these labels will only be used for visualization purposes.
%You could generate similar data files yourself using code from Tutorial 2.

%\opt{soln}{\newpage}
\begin{enumerate}[label=\alph*)]

%%%%%%%%%%%%%% (a) %%%%%%%%%%%%%%
\item Write code that reads in the spin configurations for the Ising model for a given lattice size and determines the principal components $x'_1, x'_2, \ldots$.
Make a scatter plot of $x'_1$ versus $x'_2$ for each of the lattice sizes.
What do you notice about the behaviour of the resulting two-dimensional cluster(s) as $L$ increases?

\hint{You may find it useful to use the function
\begin{center}
\texttt{(lamb, P) = np.linalg.eig(np.dot(Xc.T, Xc))}
\end{center}
When \texttt{np.dot(Xc.T, Xc)} is an $N \times N$ matrix, this function will return the $N$ eigenvalues $\lambda_1, \lambda_2, \ldots, \lambda_N$ in the array \texttt{lamb}.
The eigenvector corresponding to \texttt{lamb[i]} will be returned in \texttt{P[:,i]}.
}

%%% SOLUTION %%%
\soln{See \texttt{tutorial4{\textunderscore}pca{\textunderscore}solution.py} for the code needed to generate the plots.
You should get results similar to the following:
\begin{center}
\includegraphics[width=5cm]{xPrime1_xPrime2_Ising_partA_L20.pdf}
\includegraphics[width=5cm]{xPrime1_xPrime2_Ising_partA_L40.pdf}
\includegraphics[width=5cm]{xPrime1_xPrime2_Ising_partA_L80.pdf}
\end{center}
You can see that there are three clusters for each $L$, which become more distinct as $L$ increases.
}

%%%%%%%%%%%%%% (b) %%%%%%%%%%%%%%
\opt{soln}{\newpage}
\item Label the points in your plot such that they are coloured according to their temperature and compare with Figure 2 of Reference~\cite{wang}.
What does each cluster correspond to in terms of the phases of the two-dimensional Ising model?

%%% SOLUTION %%%
\soln{See \texttt{tutorial4{\textunderscore}pca{\textunderscore}solution.py} for the code needed to generate the plots.
You should get results similar to the following:
\begin{center}
\includegraphics[width=5cm]{xPrime1_xPrime2_Ising_L20.pdf}
\includegraphics[width=5cm]{xPrime1_xPrime2_Ising_L40.pdf}
\includegraphics[width=5cm]{xPrime1_xPrime2_Ising_L80.pdf}
\end{center}
The red cluster in the middle corresponds to the high-temperature (paramagnetic) phase. 
The white/blue clusters to the right and left of of the red cluster correspond to the low-temperature (ferromagnetic) phase
(one for the spin-up symmetry-broken state and one for the spin-down state).
}

%%%%%%%%%%%%%% (c) %%%%%%%%%%%%%%
\item Consider now the explained variance ratios $r_\ell$. 
Plot the largest 10 explained variance ratios for each lattice size and compare with Figure 1 of Reference~\cite{wang}.
How many principal components are needed to explain how the Ising spin configurations vary as a function of temperature?

%%% SOLUTION %%%
\soln{See \texttt{tutorial4{\textunderscore}pca{\textunderscore}solution.py} for the code needed to generate the plot below.
You should get results similar to the following:
\begin{center}
\includegraphics[width=8cm]{ratios_Ising.pdf}
\end{center}
Note the logarithmic scale on the $y$-axis. 
The plot shows that, for each $L$, the explained variance ratio corresponding to the first principal component is more than an order of
magnitude larger than for the other components.
We can conclude that the first principle component explains the vast majority of the variations in the spin configurations
as a function of temperature.
}


%%%%%%%%%%%%%% (d) %%%%%%%%%%%%%%
\opt{soln}{\newpage}
\item Let $p_\ell$ be the $i^{\text{th}}$ column of the matrix $P$ such that $x'_\ell = X p_\ell$.
Plot the elements of $p_1$. 
What does your plot tell you about how $x'_1$ is computed from the data $X$? 
Relate your plot the to the magnetization order parameter for the Ising model, which is given by $\frac{1}{N} \sum_i s_i$.

%%% SOLUTION %%%
\soln{See \texttt{tutorial4{\textunderscore}pca{\textunderscore}solution.py} for the code needed to generate the plots.
You should get results similar to the following:
\begin{center}
\includegraphics[width=5cm]{p1_Ising_L20.pdf}
\includegraphics[width=5cm]{p1_Ising_L40.pdf}
\includegraphics[width=5cm]{p1_Ising_L80.pdf}
\end{center}

For each value of $L$, the fluctuations in the components of $p_1$ are very small (i.e. the distribution of the components of $p_1$ is nearly flat).
As a result, $y_1 = X p_1$ will (approximately) sum the spins in each configuration such that $y_1 \appropto m$, 
where $m = \frac{1}{N} \sum_i s_i$ is the magnetization of a given configuration.
}

%%%%%%%%%%%%%% (e) %%%%%%%%%%%%%%
%\item Repeat parts (a)--(c) for the Ising gauge theory and label the points in the $x'_1$ versus $x'_2$ plot according to the label ($T=0$ or $T=\infty$). 
%You should find that plots of $x'_1$ versus $x'_2$ form one large cluster with no clear separation of the $T=0$ data from the $T=\infty$ data.
%Consider how the explained variance ratios behave as a function of $\ell$. 
%What do these explained variance ratios indicate about the order parameter for the Ising gauge theory?

%%% SOLUTION %%%
%\soln{See \texttt{tutorial4{\textunderscore}pca{\textunderscore}solution.py} for the code needed to generate the plots.
%Plots of $y_1$ versus $y_2$ should look similar to the following, with no clear separation of the $T=0$ data (blue) from the $T=\infty$ data (red).
%\begin{center}
%\includegraphics[width=5cm]{xPrime1_xPrime2_gaugeTheory_L20.pdf}
%\includegraphics[width=5cm]{xPrime1_xPrime2_gaugeTheory_L40.pdf}
%\includegraphics[width=5cm]{xPrime1_xPrime2_gaugeTheory_L80.pdf}
%\end{center}

%\opt{soln}{\newpage}
%The explained variance ratios should behave similar to the following plot:
%\begin{center}
%\includegraphics[width=8cm]{ratios_gaugeTheory.pdf}
%\end{center}
%We see that many principle components are needed in this case in order to explain how the spin configurations vary when $T$ changes from 0 to $\infty$.
%We know that the order parameter is a highly non-linear function of our spins for the Ising gauge theory, 
%so it makes sense that the linear PCA transformation is not useful for finding a low-dimensional representation of this model's spin configurations.
%}

\end{enumerate}

\begin{thebibliography}{}

\bibitem{wang} 
L. Wang, Phys. Rev. B \textbf{94}, 195105 (2016), {\small\url{https://arxiv.org/abs/1606.00318}}.

\end{thebibliography}

\end{document}